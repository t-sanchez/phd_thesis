%!TEX root = thesis_presentation.tex
\section{Deep RL results}

\begin{frame}{Experimental setting}
    \vspace{-0.7cm}    
    \begin{columns}
        \column{0.8\textwidth}
    
    \textbf{Baselines}
    \begin{itemize}
        \item \textbf{Random}: Fixed number of center frequencies, the rest is random
        \item Low-to-high (\textbf{LtH}): Sample from low to high frequencies.
        \item \textbf{sLBCS}: Stochastic LBCS
        \item \textbf{RL}: Greedy Policy Gradient approach of \cite{bakker2020experimental}.
        \item Non-adaptive Oracle \textbf{NA Oracle}: LBCS, trained on the \textit{test} set.
    \end{itemize}
    \vfill
    \textbf{Different processing pipelines}
    \begin{itemize}
        \item $128 \times 128$ image size
        \item Horizontal vs vertical sampling masks
        \item Cropped data vs cropped+resized data 
        \item Reconstructor pretrained using different masks distributions:
        \begin{itemize}
            \item \cite{zhang2019reducing}: Covers a long range of sampling rates continuously%Covers $10\%$ to $37.5\%$ continuously
            \item \cite{bakker2020experimental}: Covers a short range of sampling rates intermittently %$12.5\%$ to $25\%$
        \end{itemize}
    \end{itemize}
    \vfill
    \textbf{Different architectures}
    \begin{itemize}
        \item UNet \parencite{ronneberger2015u,zbontarFastMRIOpenDataset2019}
        \item cResNet (Cascade of ResNets) \parencite{zhang2019reducing}
    \end{itemize}
    \column{0.18\textwidth}
    \centering 
    Cropped data
    \includegraphics[width=\linewidth]{figs/image_cropped.pdf}

    \vspace{.5cm}
    Cropped+resized data
    \includegraphics[width=\linewidth]{figs/image_resized_cropped.pdf}
    %ToDo: Image of cropped vs cropped+resized
    \end{columns}
    %- Present the baselines, random, lth and put maybe a gif of the evolving masks with added samples.
    %- Illustrate also the different setting, cvb, cvz, c+rhz.
    %- Keep as a backup slide the specific differences and methodology that was deployed throughout. 
    %-Illustrate in the appendix both architectures used
\end{frame}

\begin{frame}{Implementation details}
    \begin{itemize}
        \item \textbf{Mask regimes.}
        \begin{itemize}
            \item \textbf{\cite{bakker2020experimental}.} Sampling rates: ($25\%$,$25\%$,$25\%$, $16.7\%$,$16.7\%$, and $12.5\%$); Center freq. ($25\%$,$16.7\%$,$12.5\%$, $16.7\%$,$12.5\%$,$12.5\%$)
            \item \textbf{\cite{zhang2019reducing}} Sampling rates: uniformly between $0$ and $38$; center freq. $10$ lines.
        \end{itemize}
        \item UNet trained following \cite{zbontarFastMRIOpenDataset2019} (except for LR kept constant.)
        \item cResNet: Adam, LR=$10^{-3}$, $\beta=(0.9,999)$. 
        \item 50 epochs, batch size $32$
        \item \textbf{Hardware.} DGX-2 server using $A100$ GPUs.
    \end{itemize}
\end{frame}

\begin{frame}{Data processing differences with \cite{bakker2020experimental}}
    \begin{enumerate}
        \item \textbf{Train-test split:} Different train-test split: $10\%$ of the training set as a test set, fastMRI validation set used for test.\\
        $50\%$ more central slices: $15599$ training, $1743$ validation, $3564$ test.
        \item \textbf{Complex data undersampling:} Complex preprocessing of data, followed by taking magnitude of the observation obtained \textit{after} undersampling. 
        \item \textbf{Data range:} Use the maximal intensity of the \textit{slice} instead of \textit{volume} for PSNR/SSIM computations
        \item \textbf{Data standardization:} Data standardization using \textit{observation} statistics only.
    \end{enumerate}

\end{frame}

\begin{frame}{Data processing differences with \cite{bakker2020experimental}}{Impact on the metrics}
    AUC on test, using SSIM.
    Mismatched = trained in the setting of \cite{bakker2020experimental}, but evaluated complex data preprocessing instead of magnitude data preprocessing
        \begin{center}
        \resizebox{\textwidth}{!}{\begin{tabular}{lccc|cc}
\toprule

\multirow{2}{*}{\textbf{Policy}} & \textbf{Bakker's } & \textbf{Individual } & \textbf{Matched } & \textbf{Mismatched} & \textbf{Complex preprocessing} \\
& \textbf{setting} & \textbf{data range}& \textbf{standardization}& \textbf{setting}&  \textbf{(matched - ours)}\\
\midrule
\textbf{Random (SH)} & \textit{0.6251} & 0.5964 & 0.5798 & 0.5595 & 0.5608 \\
\textbf{LtH (SH)} & 0.6073 & 0.5762 & 0.5650 & 0.5617 & 0.5719 \\
\textbf{LBCS (SH)} & 0.6375 & 0.6099 & 0.5928 & 0.5684 & 0.5732 \\
\textbf{RL (SH)} & \textbf{0.6388} & \textbf{0.6112} & \textit{\textbf{0.5941}} & \textbf{0.5698} & \textbf{0.5738} \\
\midrule
\textbf{NA Oracle (SH)} & 0.6383 & 0.6107 & 0.5937 & 0.5717 & 0.5738\\
\bottomrule
\end{tabular}} 
        \end{center}
\end{frame}

\begin{frame}{Exact reproduction of Bakker et al.}{Impact on the metrics}
Using the same train-val-test split as Bakker et al.
\begin{center}
    \begin{tabular}{lccc}
    \toprule
\textbf{Policy (SH)} & \textbf{Our code} & \textbf{Bakker} & \textbf{Difference}\\
\midrule
\textbf{Random} & 0.6974 & 0.6948 & \textbf{+0.0026}\\
\textbf{Equi (one)} & 0.7078 & 0.7049  & \textbf{+0.0029}\\
\textbf{Equi (two)} & 0.7090 & 0.7064  & \textbf{+0.0026}\\
\textbf{NA Oracle} & 0.7235 & 0.7213 & \textbf{+0.0022}\\
%\textbf{LtH (SH)} & 0.6587 & 0.6326 & 0.6218 & 0.6183 & 0.6267 \\
%\textbf{LBCS (SH)} & 0.7146 & 0.6941 & 0.6732 & 0.6317 & 0.6288 \\
\textbf{LBCS} & 0.7222 & --  & --\\
\textbf{RL (Greedy)} & 0.7250 & 0.7230 & \textbf{+0.0020}\\
\textbf{RL $\mathbf{\gamma=0.9}$} & 0.7250 & 0.7232 & \textbf{+0.0018}\\
\bottomrule
\end{tabular}
\end{center}

Consistent with the results displayed in the paper of \cite{bakker2020experimental}, although this is not an \textit{exact} replication.
\end{frame}

\begin{frame}{Results on Bakker et al. (2020)}
        \begin{itemize}
            \item Area Under Curve (AUC)
        \end{itemize}
        \begin{center}
        \resizebox{0.75\linewidth}{!}{\begin{tabular}{lcccc}
    \toprule
    
   \multirow{2}{*}{\textbf{Policy}}& \multicolumn{2}{c}{\texttt{cropped, vert., Bakker}} & \multicolumn{2}{c}{\texttt{cropped+resized, horiz., Zhang}}\\
   \cmidrule(l){2-3} \cmidrule(l){4-5}
   & UNet  & cResNet & UNet  & cResNet \\
   \midrule
   \textbf{Random}  &  $0.4348\pm 0.0001$  & $0.4432\pm 0.0002$ &$0.5860\pm 0.0002$&$0.5885\pm 0.0002$ \\
   \textbf{LtH}  & $0.4849$                & $0.5107$&$0.6678$ &$0.6902$ \\
   \textbf{LBCS}  & $0.5134\pm0.0004$      & $\mathbf{0.5243\pm 0.0003}$& $\mathbf{0.7035\pm 0.0001}$ &$0.7096 \pm 0.0001$\\
   \textbf{RL} & $\mathbf{0.5142\pm 0.0001}$        & $\mathbf{0.5242\pm0 .0002}$ &$\mathbf{0.7035 \pm 0.0002}$&$\mathbf{0.7111\pm 0.0009}$\\
   \midrule
   \textbf{NA Oracle}  & $0.5140$          & $0.5247$          & $0.7038$    &$0.7099$\\
   
   \bottomrule
   \end{tabular}
   
   }%Explicit path b.c. it is not covered by graphicspath
        \end{center}
  \end{frame}

\begin{frame}{Additional results on Bakker et al. (2020)}{Different mask regimes (cropped + vertical)}
    \begin{itemize}
        \item Consider different sampling \textit{horizons}.
        \begin{itemize}
            \item Short horizon: from $16$ to $32$ lines.
            \item Long horizon: from $4$ to $32$ lines.
        \end{itemize}
        \item Limited return on investment of adopting RL over LBCS:
        \begin{itemize}
            \item Moving from UNet to cResNet:  $\sim 0.01$ SSIM improvement, one order of magnitude larger than what RL brings over LBCS (around $0.0015$ at best).
            \item Distribution of masks matters: Bakker for SH, Zhang for LH
        \end{itemize}
    \end{itemize}

    \begin{center}
    \resizebox{0.8\linewidth}{!}{
    \begin{tabular}{lcccccccc}
        \toprule
        \multirow{3}{*}{\textbf{Policy}} & \multicolumn{4}{c}{\textbf{Short Horizon}} & \multicolumn{4}{c}{\textbf{Long Horizon}} \\
        \cmidrule(l){2-5}\cmidrule(l){6-9}
         & \multicolumn{2}{c}{\texttt{Bakker mask}} & \multicolumn{2}{c}{\texttt{Zhang mask}} & \multicolumn{2}{c}{\texttt{Bakker mask}} & \multicolumn{2}{c}{\texttt{Zhang mask}} \\
          \cmidrule(l){2-3}\cmidrule(l){4-5}\cmidrule(l){6-7}\cmidrule(l){8-9}
         & \multicolumn{1}{c}{{UNet}} & \multicolumn{1}{c}{{cResNet}} & \multicolumn{1}{c}{{UNet}} & \multicolumn{1}{c}{{cResNet}} & \multicolumn{1}{c}{{UNet}} & \multicolumn{1}{c}{{cResNet}} & \multicolumn{1}{c}{{UNet}} & \multicolumn{1}{c}{{cResNet}} \\
        \midrule
        \textbf{Random}    & {$0.5608$}            & {$0.5659$}  & {$0.5580$} & { $0.5661$}  &  $0.4348$ & $0.4432$&$0.4483$& $0.4621$\\
        \textbf{LtH}         & {$0.5719$}            & $0.5764$         & $0.5663$    & $0.5759$   & $0.4849$   & $0.5107$&$0.5122$ &$0.5218$      \\
        \textbf{LBCS}        &{$0.5732$}            & $0.5828$           & {$0.5712$} & {$0.5822$} & $0.5134$& $\mathbf{0.5243}$&{$0.5174$} &$0.5328$\\
        \textbf{RL }         & $\mathbf{0.5738}$            & $\mathbf{0.5830}$   & $\mathbf{0.5717}$ & $\mathbf{0.5829}$ &$\mathbf{0.5142}$ & $\mathbf{0.5242}$ &$\mathbf{0.5183}$&$\mathbf{0.5334}$ \\
        \midrule
        \textbf{NA Oracle}          & $0.5738$ & {$0.5832$}  & $0.5718$          & $0.5835$& $0.5140$& $0.5247$  & $0.5180$ &$0.5336$\\
        \bottomrule
        \end{tabular}}
    \end{center}
\end{frame}


\begin{frame}{Additional results on Bakker et al.}{SSIM AUC}
    \begin{center}
        \resizebox{0.8\linewidth}{!}{\begin{tabular}{lcccc|cccc}
\toprule
\multirow{3}{*}{\textbf{Policy}} & \multicolumn{4}{c|}{\textbf{Short Horizon}} & \multicolumn{4}{c}{\textbf{Long Horizon}} \\
\cmidrule(l){2-5}\cmidrule(l){6-9}
 & \multicolumn{2}{c}{\texttt{cvb}} & \multicolumn{2}{c|}{\texttt{c+rhz}} & \multicolumn{2}{c}{\texttt{cvb}} & \multicolumn{2}{c}{\texttt{c+rhz}} \\
  \cmidrule(l){2-3}\cmidrule(l){4-5}\cmidrule(l){6-7}\cmidrule(l){8-9}
 & \multicolumn{1}{c}{\textbf{UNet}} & \multicolumn{1}{c}{\textbf{CResNet}} & \multicolumn{1}{c}{\textbf{UNet}} & \multicolumn{1}{c|}{\textbf{CResNet}} & \multicolumn{1}{c}{\textbf{UNet}} & \multicolumn{1}{c}{\textbf{CResNet}} & \multicolumn{1}{c}{\textbf{UNet}} & \multicolumn{1}{c}{\textbf{CResNet}} \\
\midrule
\textbf{Random}    & {0.5608}            & {0.5659}  & {0.7292} & {0.7338}  &  0.4348 & 0.4432&0.5860& 0.5885\\
\textbf{LtH}         & {0.5719}            & 0.5764         & 0.7309      & 0.7412   & 0.4849   & 0.5107&0.6678 &0.6902      \\
\textbf{LBCS}        &{0.5732}            & 0.5828           & \textbf{0.7430} & \textbf{0.7526} & 0.5134& \textbf{0.5243}&\textbf{0.7035} &0.7096 \\
\textbf{RL }         & \textbf{0.5738}            & \textbf{0.5830}   & \textbf{0.7430} & 0.7524 &\textbf{0.5142} & \textbf{0.5242} &\textbf{0.7035}&\textbf{0.7111} \\
\midrule
\textbf{NA Oracle}          & \multicolumn{1}{l}{0.5738} & {0.5832}  & 0.7430          & 0.7527& 0.5140& 0.5247  & 0.7038 &0.7099\\[2mm]
\bottomrule
\toprule
\multirow{3}{*}{\textbf{Policy}} & \multicolumn{4}{c|}{\textbf{Short Horizon}} & \multicolumn{4}{c}{\textbf{Long Horizon}} \\
\cmidrule(l){2-5}\cmidrule(l){6-9}
 & \multicolumn{2}{c}{\texttt{cvz}} & \multicolumn{2}{c|}{\texttt{c+rvz}} & \multicolumn{2}{c}{\texttt{cvz}} & \multicolumn{2}{c}{\texttt{c+rvz}} \\
 \cmidrule(l){2-3}\cmidrule(l){4-5}\cmidrule(l){6-7}\cmidrule(l){8-9}
 & \multicolumn{1}{c}{\textbf{UNet}} & \multicolumn{1}{c}{\textbf{CResNet}} & \multicolumn{1}{c}{\textbf{UNet}} & \multicolumn{1}{c|}{\textbf{CResNet}} & \multicolumn{1}{c}{\textbf{UNet}} & \multicolumn{1}{c}{\textbf{CResNet}} & \multicolumn{1}{c}{\textbf{UNet}} & \multicolumn{1}{c}{\textbf{CResNet}} \\
 \midrule
\textbf{Random } & {0.5580} & 0.5661 & 0.6851 & 0.6984 & 0.4483 & 0.4621 & 0.5252 & 0.5276 \\
\textbf{LtH } & {0.5663} & 0.5759 & {0.6739} & {0.7028} & 0.5122 & 0.5218 & 0.6292 & 0.6444 \\
\textbf{LBCS } & {0.5712} & {0.5822} & {0.7034} & 0.7235 & 0.5174 & 0.5328 & 0.6568 & 0.6712 \\
\textbf{RL } & \textbf{0.5717} & \textbf{0.5829} & \textbf{0.7042} & \textbf{0.7239} & \textbf{0.5183} & \textbf{0.5334} & \textbf{0.6582} & \textbf{0.6733} \\
\midrule
\textbf{NA Oracle } & 0.5718 & 0.5835 & 0.7038 & 0.7236 & 0.5180 & 0.5336 & 0.6568 & 0.6717\\
\bottomrule
\end{tabular}}
    \end{center}
\end{frame}

\begin{frame}{Additional results on Bakker et al.}{SSIM at $25\%$}
    \begin{center}
        \resizebox{0.8\linewidth}{!}{\begin{tabular}{lcccc|cccc}
\toprule
\multirow{3}{*}{\textbf{Policy}} & \multicolumn{4}{c|}{\textbf{Short Horizon}} & \multicolumn{4}{c}{\textbf{Long Horizon}} \\
\cmidrule(l){2-5}\cmidrule(l){6-9}
 & \multicolumn{2}{c}{\texttt{cvb}} & \multicolumn{2}{c|}{\texttt{c+rhz}} & \multicolumn{2}{c}{\texttt{cvb}} & \multicolumn{2}{c}{\texttt{c+rhz}} \\
 \cmidrule(l){2-3}\cmidrule(l){4-5}\cmidrule(l){6-7}\cmidrule(l){8-9}
 & \textbf{UNet} & \textbf{CResNet} & \textbf{UNet} & \textbf{CResNet} & \textbf{UNet} & \textbf{CResNet} & \textbf{UNet} & \textbf{CResNet} \\
 \midrule
\textbf{Random} & {0.607} & 0.6141 & 0.7565 & 0.7667 & 0.5249 & 0.5432 & 0.6567 & 0.6567 \\
\textbf{LtH } & {0.6267} & 0.6313 & {0.7602} & {0.7786} & 0.5832 & 0.6197 & 0.7325 & 0.7714 \\
\textbf{LBCS } & {0.6288} & {0.6413} & \textbf{0.7751} & \textbf{0.7923} & 0.6294 & \textbf{0.6417} & \textbf{0.7768} & \textbf{0.7941} \\
\textbf{RL } & \textbf{0.6298} & \textbf{0.6417} & {0.7745} & \textbf{0.7921} & \textbf{0.6298} & \textbf{0.6415} & {0.7761} & {0.7935} \\
\midrule
\textbf{NA Oracle } & 0.6301 & 0.6421 & 0.7751 & 0.7926 & 0.6301 & 0.6428 & 0.7771 & 0.7942\\[2mm]
\bottomrule
\toprule
\multirow{3}{*}{\textbf{Policy}} & \multicolumn{4}{c|}{\textbf{Short Horizon}} & \multicolumn{4}{c}{\textbf{Long Horizon}} \\
\cmidrule(l){2-5}\cmidrule(l){6-9}
 & \multicolumn{2}{c}{\texttt{cvz}} & \multicolumn{2}{c|}{\texttt{c+rvz}} & \multicolumn{2}{c}{\texttt{cvz}} & \multicolumn{2}{c}{\texttt{c+rvz}} \\
 \cmidrule(l){2-3}\cmidrule(l){4-5}\cmidrule(l){6-7}\cmidrule(l){8-9}
 & \textbf{UNet} & \textbf{CResNet} & \textbf{UNet} & \textbf{CResNet} & \textbf{UNet} & \textbf{CResNet} & \textbf{UNet} & \textbf{CResNet} \\
 \midrule
\textbf{Random } & {0.6053} & 0.6148 & 0.7201 & 0.7401 & 0.5502 & 0.5693 & 0.6191 & 0.6359 \\
\textbf{LtH } & {0.619} & 0.6309 & {0.7052} & {0.7487} & 0.6190 & 0.6309 & 0.7052 & 0.7487 \\
\textbf{LBCS } & {0.6282} & {0.6413} & {0.7411} & {0.7710} & 0.6293 & \textbf{0.6451} & {0.7453} & \textbf{0.7746} \\
\textbf{RL } & \textbf{0.6288} & \textbf{0.6416} & \textbf{0.7417} & \textbf{0.7714} & \textbf{0.6304} & \textbf{0.6452} & \textbf{0.7470} & {0.7738} \\
\midrule
\textbf{NA Oracle } & 0.6292 & {0.643} & 0.741 & 0.7717 & {0.6304} & {0.6464} & 0.7457 & 0.7755\\
\bottomrule
\end{tabular}

}
        \end{center}
\end{frame}

\begin{frame}{Additional results on Bakker et al.}{PSNR AUC}
    \begin{center}
        \resizebox{.8\textwidth}{!}{\begin{tabular}{lcccc|cccc}
\toprule
\multirow{3}{*}{\textbf{Policy}} & \multicolumn{4}{c|}{\textbf{Short Horizon}} & \multicolumn{4}{c}{\textbf{Long Horizon}} \\
\cmidrule(l){2-5}\cmidrule(l){6-9}
 & \multicolumn{2}{c}{\texttt{cvb}} & \multicolumn{2}{c|}{\texttt{c+rhz}} & \multicolumn{2}{c}{\texttt{cvb}} & \multicolumn{2}{c}{\texttt{c+rhz}} \\
  \cmidrule(l){2-3}\cmidrule(l){4-5}\cmidrule(l){6-7}\cmidrule(l){8-9}
 & \multicolumn{1}{c}{\textbf{UNet}} & \multicolumn{1}{c}{\textbf{CResNet}} & \multicolumn{1}{c}{\textbf{UNet}} & \multicolumn{1}{c|}{\textbf{CResNet}} & \multicolumn{1}{c}{\textbf{UNet}} & \multicolumn{1}{c}{\textbf{CResNet}} & \multicolumn{1}{c}{\textbf{UNet}} & \multicolumn{1}{c}{\textbf{CResNet}} \\
\midrule
\textbf{Random}    & {24.206}            & {24.180}  & {27.641} & {28.365}  &  20.802 & 20.836 &22.375& 22.371\\
\textbf{LtH}         & {24.486}            & 24.420         & 27.628      & 28.640   & 23.570   & 23.578&26.607 &27.317      \\
\textbf{LBCS}        &\textbf{24.493}            & \textbf{24.517}           & \textbf{28.254} & \textbf{29.154} & 23.574& {23.635}&\textbf{27.220} &\textbf{27.802} \\
\textbf{RL }         & {24.466}            & {24.506}   & {28.247} & 29.147 &\textbf{23.585} & \textbf{23.646} &{27.210}&{27.757} \\
\midrule
\textbf{NA Oracle}          & 24.510 & {24.521}  &28.257         & 29.152& 23.591 & 23.639  &27.222 &27.757\\[2mm]
\bottomrule
\toprule
\multirow{3}{*}{\textbf{Policy}} & \multicolumn{4}{c|}{\textbf{Short Horizon}} & \multicolumn{4}{c}{\textbf{Long Horizon}} \\
\cmidrule(l){2-5}\cmidrule(l){6-9}
 & \multicolumn{2}{c}{\texttt{cvz}} & \multicolumn{2}{c|}{\texttt{c+rvz}} & \multicolumn{2}{c}{\texttt{cvz}} & \multicolumn{2}{c}{\texttt{c+rvz}} \\
 \cmidrule(l){2-3}\cmidrule(l){4-5}\cmidrule(l){6-7}\cmidrule(l){8-9}
 & \multicolumn{1}{c}{\textbf{UNet}} & \multicolumn{1}{c}{\textbf{CResNet}} & \multicolumn{1}{c}{\textbf{UNet}} & \multicolumn{1}{c|}{\textbf{CResNet}} & \multicolumn{1}{c}{\textbf{UNet}} & \multicolumn{1}{c}{\textbf{CResNet}} & \multicolumn{1}{c}{\textbf{UNet}} & \multicolumn{1}{c}{\textbf{CResNet}} \\
 \midrule
\textbf{Random } & {23.978} & 24.184 & 25.845 & 26.949 & 20.967 & 21.437 & 20.428 & 20.625 \\
\textbf{LtH } & {24.108} & 24.396 & {25.641} & {27.048} & 23.441 & 23.703 & 24.487 & 25.510 \\
\textbf{LBCS } & \textbf{24.269} & {24.490} & {26.515} & \textbf{27.870} & \textbf{23.600} & 23.846 & \textbf{25.307} & \textbf{26.316} \\
\textbf{RL } & 24.226 & \textbf{24.498} & \textbf{26.530} & {27.861} & {23.581} & \textbf{23.858} & {25.240} & {26.246} \\
\midrule
\textbf{NA Oracle } & 24.271 & 24.513 & 26.528 & 27.856 & 23.603 & 23.848 & 25.309 & 26.250\\
\bottomrule
\end{tabular}}
    \end{center}
\end{frame}

\begin{frame}{Additional results on Bakker et al.}{PSNR at $25\%$}
    \begin{center}
        \resizebox{.8\textwidth}{!}{\begin{tabular}{lcccc|cccc}
\toprule
\multirow{3}{*}{\textbf{Policy}} & \multicolumn{4}{c|}{\textbf{Short Horizon}} & \multicolumn{4}{c}{\textbf{Long Horizon}} \\
\cmidrule(l){2-5}\cmidrule(l){6-9}
 & \multicolumn{2}{c}{\texttt{cvb}} & \multicolumn{2}{c|}{\texttt{c+rhz}} & \multicolumn{2}{c}{\texttt{cvb}} & \multicolumn{2}{c}{\texttt{c+rhz}} \\
 \cmidrule(l){2-3}\cmidrule(l){4-5}\cmidrule(l){6-7}\cmidrule(l){8-9}
 & \textbf{UNet} & \textbf{CResNet} & \textbf{UNet} & \textbf{CResNet} & \textbf{UNet} & \textbf{CResNet} & \textbf{UNet} & \textbf{CResNet} \\
 \midrule
\textbf{Random} & {24.583} & 24.604 & 28.241 & 29.113 & 21.520 & 21.684 & 23.547 & 23.843 \\
\textbf{LtH } & {25.056} & 24.984 & {28.176} & {29.567} & 25.056 & 24.984 & 28.176 & 29.567 \\
\textbf{LBCS } & \textbf{25.068} & \textbf{25.117} & \textbf{29.088} & {30.214} & 25.043 & {25.077} & \textbf{29.226} & \textbf{30.234} \\
\textbf{RL } & {25.032} & {25.108} & {29.050} & \textbf{30.218} & \textbf{25.071} & \textbf{25.101} & {29.194} & {30.188} \\
\midrule
\textbf{NA Oracle } & 25.095 & 25.120 & 29.090 & 30.233 & 25.111 & 25.111 & 29.211 & 30.229\\[2mm]
\bottomrule
\toprule
\multirow{3}{*}{\textbf{Policy}} & \multicolumn{4}{c|}{\textbf{Short Horizon}} & \multicolumn{4}{c}{\textbf{Long Horizon}} \\
\cmidrule(l){2-5}\cmidrule(l){6-9}
 & \multicolumn{2}{c}{\texttt{cvz}} & \multicolumn{2}{c|}{\texttt{c+rvz}} & \multicolumn{2}{c}{\texttt{cvz}} & \multicolumn{2}{c}{\texttt{c+rvz}} \\
 \cmidrule(l){2-3}\cmidrule(l){4-5}\cmidrule(l){6-7}\cmidrule(l){8-9}
 & \textbf{UNet} & \textbf{CResNet} & \textbf{UNet} & \textbf{CResNet} & \textbf{UNet} & \textbf{CResNet} & \textbf{UNet} & \textbf{CResNet} \\
 \midrule
\textbf{Random } & {24.369} & 24.623 & 26.657 & 27.957 & 21.791 & 22.497 & 21.854 & 22.506 \\
\textbf{LtH } & {24.551} & 24.961 & {26.223} & {28.143} & 24.551 & 24.961 & 26.223 & 28.143 \\
\textbf{LBCS } & \textbf{24.817} & {25.089} & {27.439} & \textbf{29.069} & 24.850 & \textbf{25.084} & {27.518} & \textbf{29.145} \\
\textbf{RL } & {24.765} & \textbf{25.102} & \textbf{27.451} &  \textbf{29.069} &\textbf{25.102} & {24.807} & \textbf{27.583} & {29.041} \\
\midrule
\textbf{NA Oracle } & 24.813 & {25.123} & 27.429 & 29.043 & {24.837} & {25.101} & 27.446 & 29.072\\
\bottomrule
\end{tabular}

}
    \end{center}
\end{frame}


\begin{frame}{Pineda results}
    \begin{minipage}[c]{0.57\linewidth}
        \resizebox*{\linewidth}{!}{
    \begin{tabular}{lcccccc}
        \toprule
       \multirow{2}{*}{\textbf{Policy}}& \multicolumn{3}{c}{\textbf{SSIM}}& \multicolumn{3}{c}{\textbf{PSNR}}\\
       \cmidrule(l){2-4}\cmidrule(l){5-7}
       & Samp. rate & Acc. factor & Final rate & Samp. rate & Acc. factor & Final rate\\
       \midrule
       \textbf{Random} & 0.5801&0.4497&0.6723 &  26.489 &22.327&28.962\\
       %\textbf{Random low bias} & $0.6010$ & $27.817$ \\
       \textbf{LtH} & 0.5636 &0.4506&0.6686& 27.169&23.133&29.360\\
       \textbf{LBCS} & 0.6079 &0.4787&\textbf{0.6886}& \textbf{28.491}&23.799&\textbf{ 30.211}\\
       %\textbf{Evaluator} & 0.5933&0.4671&0.6812& 27.665&23.2691&29.681 \\
       \textbf{DS-DDQN} & 0.6101&\textbf{0.4797}&0.6855&28.240&\textbf{23.978}&29.652 \\
       \textbf{SS-DDQN} & \textbf{0.6139}&\textbf{0.4797}& 0.6882&28.424 &23.918&29.929\\
       \midrule
       \textbf{Adaptive Oracle} & 0.6341 &0.4910&0.7131& 29.013&24.498&30.683\\
       
       \bottomrule
       \end{tabular}}
    \end{minipage}
    \begin{minipage}[c]{0.4\linewidth}
        \includegraphics[width=\linewidth]{simple_baselines/pineda_plots/pineda_psnr_accellFalse.pdf}
    \end{minipage}
\end{frame}

\begin{frame}{Pineda results}
    SS-DDQN variability and comparison with the LBCS mask. CoV stands for Coefficient of Variation here.
        \begin{center}
            \includegraphics[width=\linewidth]{simple_baselines/ss-ddqn_lbcs_masks.pdf}    
        \end{center}
\end{frame}


\begin{frame}{Bakker visual results}{c+rhz recon ($25\%$ and $12.5\%$)}
    \centering
    \begin{minipage}[c]{0.46\linewidth}
        \includegraphics[width=\linewidth]{simple_baselines/plot_idx_1684_acc_4.pdf}    
    \end{minipage}
    \hfill
    \begin{minipage}[c]{0.46\linewidth}
        \includegraphics[width=\linewidth]{simple_baselines/plot_idx_1684_acc_8_small.pdf}    
    \end{minipage}
\end{frame}

\begin{frame}{Bakker visual results}{recon at diff SR -- PDFS and PD images}
    \centering
    \includegraphics[width=.48\linewidth]{simple_baselines/plt_recon_full_53.pdf}
    \hfill
    \includegraphics[width=.48\linewidth]{simple_baselines/plt_recon_full_82.pdf}
\end{frame}

\begin{frame}{Additional discussion}
    \begin{itemize}
        \item What kind of metrics should be used?
        \item What about joint training? 
        \begin{itemize}
            \item Fixed sampling  \parencite{bahadir2019learning, huijben2020learning}
            \item Sequential sampling \parencite{yin2021end,van2021active,jin2019self}.
            \item What does matter? Curriculum for training recon or gain due to policy?
        \end{itemize}
        \item Sequential vs fixed sampling rate?
        \item Practical advice
        \begin{mdframed}
        \begin{enumerate}
            \item Focus on improvements in the reconstructor architecture, mask distribution and algorithms used for training the reconstructor.
            \item Compare against strong baselines, such as LBCS.
            \item Show sampling curves and use AUC to aggregate your results instead of performance at the final sampling rate.
            \item Be mindful about preprocessing settings when evaluating a policy model. We recommend using the cropped+vertical setting with the data normalization implemented by \cite{zbontarFastMRIOpenDataset2019}.
        \end{enumerate}
        \end{mdframed}
        %\item RL is known to be sensitive to variations in parameters.  \cite{Engstrom2020Implementation,ingredientsdeeppolicy} showed that the improvements of SotA RL methods can be traced to the exploitation of a subset of implementation details and algorithmic improvements and that even simple algorithms leveraging this subset can achieve SotA performance, similar to our work.
    \end{itemize}
\end{frame}