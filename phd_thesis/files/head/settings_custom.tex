%%%%%%%%%%%%%%%%%%%%%%%%%%%%%%%%%%%%%%%%%%%%%%
%
%		Thesis Settings
%		Custom settings
%
%		2011
%
%%%%%%%%%%%%%%%%%%%%%%%%%%%%%%%%%%%%%%%%%%%%%%

% %
% %   Use this file for your own custom packages, command-definitions, etc...
% %
% % 
% % Packages for references - cleverref must be last
% \usepackage{nameref}
% \usepackage{hyperref}
% \usepackage{cleveref}
% \usepackage[shortlabels]{enumitem}
% % Reduce spacing in bibliography
% \setlength{\bibsep}{0pt plus 0.3ex}
% % Allow equations to break between pages
% \allowdisplaybreaks
% % Penalty for widow and orphan
% \widowpenalty=9999
% \clubpenalty=9999
% %Penalty for relation and binary operation breaks in equations
% \relpenalty=9999
% \binoppenalty=9999
\usepackage{amsthm}
\usepackage{amsmath, bm}
\usepackage{amssymb}
\usepackage{mathrsfs}
\usepackage{mathtools}
\usepackage{graphicx}
\usepackage{multicol}
\usepackage{pgfplots}
\usepackage{changepage}
\usepackage[font={small, it}]{caption}
\usepackage{subcaption}
\usepackage{url}
\usepackage{siunitx}
\pgfplotsset{compat=newest}
%% the following commands are needed for some matlab2tikz features
\usetikzlibrary{plotmarks}
\usetikzlibrary{arrows.meta}
\usetikzlibrary{spy}
\usepgfplotslibrary{patchplots}
\usepackage{grffile}
\usepackage{nccmath}
\usepackage{array}
\usepackage[export]{adjustbox}
\usepackage{multirow}
\usepackage{setspace}
\usepackage{algorithm} 
\usepackage{algpseudocode}  
\usepackage{diagbox}
\usepackage{bbm} % For the \mathbbm{1} -> Indicator function 1.
\usepackage[autostyle]{csquotes}  
%\usepackage[algo2e,linesnumbered, ruled]{algorithm2e} 

\usepackage{xspace}
\usepackage{amsfonts}       % blackboard math symbols
\usepackage{nicefrac}       % compact symbols for 1/2, etc.
\usepackage{microtype}      % microtypography
\usepackage[acronym]{glossaries}
% Optional math commands from https://github.com/goodfeli/dlbook_notation.
\usepackage{mathrsfs,oldgerm}
\usepackage{wrapfig}
\usepackage{cleveref}
\usepackage{verbatim}
\usepackage[disable]{todonotes} %[disable] to disable todonotes
\usepackage[export]{adjustbox}
\usepackage{pifont}
%\usepackage[nonatbib]{neurips_2021}
\usepackage{placeins}       % professional-quality tables
\newcommand{\cmark}{\ding{51}}%
\newcommand{\xmark}{\ding{55}}%


\newcommand{\todoi}[1]{\todo[inline]{#1}}
\newcommand{\todot}[1]{\todo[inline,backgroundcolor=yellow]{#1}}


\usepackage{mdframed}
\usepackage{xcolor}
\setcounter{secnumdepth}{2} % Depth of numbering

\theoremstyle{definition}
\newtheorem*{definition}{Definition}

%\theorembodyfont{\normalfont}
\theoremstyle{definition}
\newtheorem*{remark}{Remark}
\surroundwithmdframed[linewidth = 0 ,%
leftmargin = 20,%
rightmargin = 20,%
backgroundcolor=black!10,%
innertopmargin = 10,%
splittopskip = 0,%
ntheorem]{remark}

% Frame around definition.
\surroundwithmdframed[linewidth = 0 ,%
leftmargin = 20,%
rightmargin = 20,%
backgroundcolor=black!10,%
innertopmargin = 10,%
splittopskip = 0,%
ntheorem]{definition}


% For the caveats in RL chapter
\newtheoremstyle{bfnoteonly}%
{}{}%
{}{}%
{\itshape}{.}%
{ }%
{\thmnote{#3}}
\theoremstyle{bfnoteonly}
\newtheorem*{extremark}{}
\surroundwithmdframed[linewidth = 0 ,%
leftmargin = 20,%
rightmargin = 20,%
backgroundcolor=black!10,%
innertopmargin = 10,%
skipabove=20,
ntheorem]{extremark}


% Define a frame aroung a proposition
\theoremstyle{theorem}
\newtheorem{proposition}{Proposition}
\surroundwithmdframed[linewidth = 0 ,%
leftmargin = 20,%
rightmargin = 20,%
backgroundcolor=black!10,%
innertopmargin = 10,%
%splittopskip=\topskip,
%skipbelow=\baselineskip,%
%skipabove=\baselineskip,%
ntheorem]{proposition}

\theoremstyle{theorem}
\newtheorem{theorem}{Theorem}
\surroundwithmdframed[linewidth = 0 ,%
leftmargin = 20,%
rightmargin = 20,%
backgroundcolor=black!10,%
innertopmargin = 10,%
%splittopskip=\topskip,
%skipbelow=\baselineskip,%
%skipabove=\baselineskip,%
ntheorem]{theorem}

\usepgfplotslibrary{fillbetween}
\pgfplotsset{every tick label/.append style={font=\scriptsize}}
\def\shortestskip{\setlength{\abovedisplayskip}{0pt}%
\setlength{\belowdisplayskip}{0pt}%
\setlength{\abovedisplayshortskip}{0pt}%
\setlength{\belowdisplayshortskip}{0pt}}

\shortestskip

\makeatletter
\newcounter{algorithmicH}% New algorithmic-like hyperref counter
\let\oldalgorithmic\algorithmic
\renewcommand{\algorithmic}{%
  \stepcounter{algorithmicH}% Step counter
  \oldalgorithmic}% Do what was always done with algorithmic environment
\renewcommand{\theHALG@line}{ALG@line.\thealgorithmicH.\arabic{ALG@line}}
\makeatother

\pgfplotsset{compat=newest,
       colormap={parula}{
            rgb255=(53,42,135)
            rgb255=(15,92,221)
            rgb255=(18,125,216)
            rgb255=(7,156,207)
            rgb255=(21,177,180)
            rgb255=(89,189,140)
            rgb255=(165,190,107)
            rgb255=(225,185,82)
            rgb255=(252,206,46)
            rgb255=(249,251,14)
        },
    }


\usepackage{booktabs}
\usepackage{soul}
\makeatletter


\def \wh{\widehat}

\def \ef{\eta(f)}
\def \emf{\eta_m(f)}
\def \emfh{\widehat{\eta}_m(f)}

\def \diag{\text{diag}}
\def\*#1{\mathbf{#1}}


\makeatletter
\newenvironment{tsubarray}[1]{%
  \vcenter\bgroup
  \Let@ \restore@math@cr \default@tag
  \baselineskip\fontdimen10 \scriptfont\tw@
  \advance\baselineskip\fontdimen12 \scriptfont\tw@
  \lineskip\thr@@\fontdimen8 \scriptfont\thr@@
  \lineskiplimit\lineskip
  \check@mathfonts
  \ialign\bgroup\ifx c#1\hfil\fi
    \normalfont\fontsize\sf@size\z@\selectfont\ignorespaces##\unskip\hfil\crcr
}{%
  \crcr\egroup\egroup
}
\makeatother
\newcommand{\tsub}[1]{\begin{tsubarray}{l}#1\end{tsubarray}}


\newcommand{\di}{\textbf{(i)} }
\newcommand{\ii}{\textbf{(ii)} }
\newcommand{\iii}{\textbf{(iii)} }
\newcommand{\iv}{\textbf{(iv)} }
\newcommand{\dv}{\textbf{(v)} }
\newcommand{\sga}{SG-v1 }
\newcommand{\sgb}{SG-v2 }
\newcommand{\ga}{G-v1 }




%\usetikzlibrary{external}
%\tikzexternalize[prefix=tikz/]

\input cyracc.def
\font\tencyr=wncysc10
\def\cyr{\tencyr\cyracc}
\def\diracComb{\mbox{\cyr SH}} 

\input{files/head/math_commands.tex}
\def \bepsilon{\boldsymbol{\epsilon}}
\def \l{\left}
\def \r{\right}
\def \R{\mathbb{R}}
\def \wh{\widehat}
\def \E{\mE}
\def \e{\ve}
\def \w{\vw}
%\def \u{\vu}
\def \v{\vv}
\def \x{\vx}
\def \y{\vy}
\def \yo{\vy_\omega}
\newcommand{\yot}[1]{\y_{omega_{#1}}}
\def \z{\mathbf{z}}

\def \PO{\mP_\Omega}
\def \Po{\mP_\omega}
\def \mAo{\mA_\omega}
\def\PXyo{\mathbf{P}_{\X|\y,\omega}}
\def \F{\mF}

\def \L{\ell}

\def \uD{\u^{\text{1D}}}

\def \X{\mX}
\def \Xh{\mathbf{\hat{X}}}
\def \Y{\mathbf{Y}}
\def \Yo{\mathbf{Y}_{\omega}}
\def \YO{\mathbf{Y}_{\Omega}}
\def \Z{\mathbf{Z}}
\def \V{\mathbf{V}}

\def \PX{\mathcal{P}_\X}
\def \PY{\mathcal{P}_\Y}

\def \PYo{\mathcal{P}_{\Yo}}
\def \JXY{\mathcal{P}_{\X,\Y}}

\def \JXYo{\mathcal{P}_{\X,\Yo}}

\def \PXY{\mathcal{P}_{\rvx|\rvy}}
\def \PYX{\mathcal{P}_{\rvy|\rvx}}
\def \pXy{p(\rvx |\vy)}
\def \pXyo{p(\rvx |\yo,\omega)}

\def \PYx{p(\Y|\x)}

\def \PXYo{\mathcal{P}_{\X|\Yo,\omega}}
\def \PYXo{\mathcal{P}_{\Yo|\X,\omega}}
\def \PXYO{\mathcal{P}_{\X|\YO,\omega}}
\def \PYXO{\mathcal{P}_{\YO|\X,\omega}}
\def \PXyo{\mathcal{P}_{\X|\yo,\omega}}
\def \PYox{\mathcal{P}_{\Yo|\x,\omega}}
\def\H{\mathcal{H}}


\def \W{\mathcal{W}}
\def \Vc{\mathcal{V}}
\def \PcO{\mathcal{P}_\Omega}
\def \Pco{\mathcal{P}_\omega}

\def \G{\mathcal{G}}
\def \P{\mathcal{P}}
\def \Q{\mathcal{Q}}
\def \PPQ{\mathcal{P}_{\mathcal{P,Q}}}
\def \PbX{\mathbb{P}_\mathcal{X}}
\def \Xc{\mathcal{X}}
\def \PbY{\mathbb{P}_\mathcal{Y}}
\def \Yc{\mathcal{Y}}
\def \Zc{\mathcal{Z}}
\def \lip{\text{Lip}}

\def \px{p(\rvx)}
\def \py{p(\y)}
\def \jxy{p(\x,\y)}
\def \pxy{p(\x|\y)}
\def \pyx{p(\y|\x)}

\def \ft{f_\theta}
\def \fts{f_{\theta^*}}

\def \ftl{f_{\theta,\lambda}}
\def \Gt{G_\theta}
\def \Gts{G_{\theta^*}}

\def \Gct{\mathcal{G}_\theta}

\def \xh{\bm{\hat{x}}}
\def \AO{Adler and \"Oktem\xspace}
\def \diag{\text{diag}}

\makeatletter
\newcommand{\bianca}{\renewcommand\NAT@open{[}\renewcommand\NAT@close{]}}
\makeatother

\newcommand*\sqcitet[1]{{\bianca\citet{#1}}}
\newcommand*\sqcitep[1]{{\bianca\citep{#1}}}

\def\bPsi{{\boldsymbol{\Psi}}}
\def\bPhi{{\boldsymbol{\Phi}}}