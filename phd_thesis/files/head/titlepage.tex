\begin{titlepage}
\begin{otherlanguage}{french}
\begin{center}
%\large
\sffamily


\null\vspace{2cm}
{\huge Learning to sample in Cartesian MRI} \\[24pt] 
\textcolor{gray}{\small{THIS IS A TEMPORARY TITLE PAGE \\ It will be replaced for the final print by a version \\ provided by the registrar's office.}}
    
\vfill

\begin{tabular} {cc}
\parbox{0.3\textwidth}{\includegraphics[width=4cm]{epfl}}
&
\parbox{0.7\textwidth}{%
	Thèse n. 9981\\
	présentée le 12 avril 2022\\
	à la Faculté des sciences et techniques de l'ingénieur\\
	Laboratoire de systèmes d'information et d'inférence\\
	Programme doctoral en informatique et communications\\
%
%	ÉCOLE POLYTECHNIQUE FÉDÉRALE DE LAUSANNE\\
	École polytechnique fédérale de Lausanne\\[6pt]
	pour l'obtention du grade de Docteur ès Sciences\\
	par\\ [4pt]
	\null \hspace{3em} Thomas Sanchez\\[9pt]
%
\small
Acceptée sur proposition du jury :\\[4pt]
%
    Prof Alexandre Alahi, président du jury\\
    Prof Volkan Cevher, directeur de thèse\\
    Prof Dimitri van de Ville, rapporteur\\
    Dr Ruud van Heeswijk, rapporteur\\
    Dr Philippe Ciuciu, rapporteur\\[12pt]
%
Lausanne, EPFL, \the\year}
\end{tabular}
\end{center}
\vspace{2cm}
\end{otherlanguage}
\end{titlepage}



